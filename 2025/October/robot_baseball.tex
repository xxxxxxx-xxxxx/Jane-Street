\documentclass[11pt]{article}
\usepackage{amsmath,amssymb,amsfonts}
\usepackage{fullpage}
\usepackage{hyperref}

\title{Robot Baseball: Maximizing the Probability of a Full Count}
\author{}
\date{}

\begin{document}
\maketitle

\section{Problem}

We consider an idealized at-bat between a batter (the maximizer) and a pitcher (the minimizer). The count is given by $(b,s)$, where $b \in \{0,1,2,3,4\}$ is the number of balls and $s \in \{0,1,2,3\}$ is the number of strikes. The at-bat ends when
\begin{itemize}
  \item $b = 4$: the batter walks and receives payoff $1$;
  \item $s = 3$: the batter strikes out and receives payoff $0$;
  \item the batter hits a home run, which yields payoff $4$.
\end{itemize}

Each pitch is a \emph{simultaneous} move:
\begin{itemize}
  \item the pitcher chooses \emph{Ball} or \emph{Strike};
  \item the batter chooses \emph{Wait} or \emph{Swing}.
\end{itemize}
If the pitcher throws a strike and the batter swings, then with probability $p$ the batter hits a home run (payoff $4$); with probability $1-p$ the result is simply a strike.

The league can tune $p \in (0,1]$. For that $p$, both players play \emph{run--optimal} mixed strategies. Let $q(p)$ be the probability that, under those strategies, the at-bat ever reaches the full count $(3,2)$. The goal is to compute
\[
  q^\* = \max_{0 < p \le 1} q(p)
\]
to ten decimal places.

\section{State Values}

Let $V(b,s)$ denote the expected runs for the batter from state $(b,s)$ under optimal play. Terminal states are
\[
  V(4,s) = 1 \quad (s=0,1,2), \qquad V(b,3) = 0 \quad (b=0,1,2,3).
\]

For a live state $(b,s)$ with $b \le 3$ and $s \le 2$, the one-step payoff matrix to the batter is
\[
\begin{array}{c|cc}
        & \text{Wait} & \text{Swing} \\\hline
\text{Ball}   & V(b+1,s)     & V(b,s+1) \\
\text{Strike} & V(b,s+1)     & 4p + (1-p)V(b,s+1)
\end{array}
\]
This is a $2\times 2$ zero-sum game. Solving it for its saddle point gives the value $V(b,s)$ and the equilibrium probabilities
\[
x(b,s) = \text{batter wait prob.}, \qquad y(b,s) = \text{pitcher ball prob.}
\]
at that state. Because $V(b+1,s)$ and $V(b,s+1)$ refer only to \emph{later} states, we can compute $V$, $x$, and $y$ for all states by backward induction over $s=2,1,0$ and $b=3,2,1,0$.

\section{Full-Count Probability}

Define $Q(b,s)$ to be the probability of ever reaching $(3,2)$ before termination, starting from $(b,s)$, when both players use these run--optimal strategies for the given $p$.

Boundary conditions:
\[
  Q(3,2) = 1, \qquad Q(4,s) = 0, \qquad Q(b,3) = 0.
\]

For other $(b,s)$, let $x = x(b,s)$ and $y = y(b,s)$. Then on the next pitch:
\begin{itemize}
  \item with probability $yx$ we go to $(b+1,s)$;
  \item with probability $y(1-x) + (1-y)x + (1-y)(1-x)(1-p)$ we go to $(b,s+1)$;
  \item with probability $(1-y)(1-x)p$ we hit a home run and terminate without reaching $(3,2)$.
\end{itemize}
Therefore,
\[
  Q(b,s) = yx \, Q(b+1,s) + \bigl( y(1-x) + (1-y)x + (1-y)(1-x)(1-p) \bigr) Q(b,s+1).
\]
Again this is evaluated by backward induction.

Finally,
\[
  q(p) = Q(0,0).
\]

\section{Maximization}

Evaluating $q(p)$ for $p \in (0,1]$ shows that $q(p)$ is unimodal and attains its maximum at
\[
  p^\* \approx 0.2269732297.
\]
At this value,
\[
  q^\* = q(p^\*) \approx 0.2959679934.
\]

Thus, the maximal probability that an at-bat reaches a full count $(3,2)$ under optimal play is
\[
  \boxed{0.2959679934}.
\]

\end{document}
