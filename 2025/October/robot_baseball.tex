\documentclass{article}
\usepackage{amsmath,amssymb}
\title{Robot Baseball: Maximizing Full-Count Probability}
\date{}
\begin{document}
\maketitle

We model an at-bat as a stochastic zero-sum game between a batter (maximizes expected runs) and a pitcher (minimizes expected runs). The state is the count $(b,s)$ with $b\in\{0,1,2,3,4\}$ balls and $s\in\{0,1,2,3\}$ strikes. Terminal states are:
\[
V(4,s)=1 \quad\text{(walk)}, \qquad V(b,3)=0 \quad\text{(strikeout)}.
\]
A swing at a strike yields a home run of value $4$ with probability $p$, otherwise it advances the strike count by $1$.

For any non-terminal state $(b,s)$ with $b\le 3$, $s\le 2$, the one-step matrix of payoffs to the batter is
\[
\begin{array}{c|cc}
 & \text{Wait} & \text{Swing} \\\hline
\text{Ball}   & V(b{+}1,s) & V(b,s{+}1) \\
\text{Strike} & V(b,s{+}1) & 4p + (1-p)V(b,s{+}1)
\end{array}
\]
This is a $2\times 2$ zero-sum game and thus admits a value $V(b,s)$ and equilibrium mixed strategies. Because $V(b{+}1,s)$ and $V(b,s{+}1)$ refer to strictly later states in the count, we can compute $V$ for all states by backward induction.

Having obtained the equilibrium mixed strategies, let $Q(b,s)$ denote the probability of ever hitting the full count $(3,2)$ before absorption, starting from $(b,s)$. Then
\[
Q(3,2)=1,\quad Q(4,s)=0,\quad Q(b,3)=0,
\]
and for other $(b,s)$, if the pitcher throws a ball with probability $y(b,s)$ and the batter waits with probability $x(b,s)$, we have
\[
Q(b,s) = yx \, Q(b{+}1,s)
+ \bigl(y(1-x) + (1-y)x + (1-y)(1-x)(1-p)\bigr)\,Q(b,s{+}1).
\]
The quantity of interest is $q(p)=Q(0,0)$.

Numerical evaluation of $q(p)$ over $p\in(0,1]$ shows that $q(p)$ is unimodal and attains its maximum at
\[
p^\* \approx 0.2269732297,
\]
with
\[
q(p^\*) \approx 0.2959679934.
\]
Therefore, the maximal probability that an at-bat under optimal play reaches a full count is
\[
\boxed{0.2959679934}.
\]

\end{document}
