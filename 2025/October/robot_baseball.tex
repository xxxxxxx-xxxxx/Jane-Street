
---

## 2) `robot_baseball.tex`

```latex
\documentclass[11pt]{article}
\usepackage{amsmath,amssymb,amsfonts}
\usepackage{fullpage}
\usepackage{hyperref}

\title{Robot Baseball: Maximizing the Probability of a Full Count}
\author{}
\date{}

\begin{document}
\maketitle

\section{Problem}

We consider an idealized at-bat between a batter (the maximizer) and a pitcher (the minimizer). The count is given by $(b,s)$, where $b \in \{0,1,2,3,4\}$ is the number of balls and $s \in \{0,1,2,3\}$ is the number of strikes. The at-bat ends when
\begin{itemize}
  \item $b = 4$: the batter walks and receives payoff $1$;
  \item $s = 3$: the batter strikes out and receives payoff $0$;
  \item the batter hits a home run, which yields payoff $4$.
\end{itemize}

Each pitch is a \emph{simultaneous} move:
\begin{itemize}
  \item the pitcher chooses \emph{Ball} or \emph{Strike};
  \item the batter chooses \emph{Wait} or \emph{Swing}.
\end{itemize}
If the pitcher throws a strike and the batter swings, then with probability $p$ the batter hits a home run (payoff $4$); with probability $1-p$ the result is simply a strike.

The league can tune $p \in (0,1]$. Both players play \emph{run--optimal} mixed strategies for the given $p$. Let $q(p)$ denote the probability that, under those optimal strategies, the at-bat ever reaches the count $(3,2)$. The problem is to compute
\[
  q^\* \;=\; \max_{0 < p \le 1} q(p)
\]
to ten decimal places.

\section{State Values}

Let $V(b,s)$ denote the value (expected runs to the batter) from state $(b,s)$ under optimal play.
Terminal states are
\[
  V(4,s) = 1 \quad (s=0,1,2), \qquad V(b,3) = 0 \quad (b=0,1,2,3).
\]

For a non-terminal state $(b,s)$ with $b \le 3$ and $s \le 2$, the one-step payoff matrix to the batter is
\[
\begin{array}{c|cc}
        & \text{Wait} & \text{Swing} \\\hline
\text{Ball}   & V(b+1,s)     & V(b,s+1) \\
\text{Strike} & V(b,s+1)     & 4p + (1-p)V(b,s+1)
\end{array}
\]
This is a $2 \times 2$ zero-sum game. Denote by $x(b,s)$ the batter's equilibrium probability of waiting, and by $y(b,s)$ the pitcher's equilibrium probability of throwing a ball. Because the two off-diagonal entries coincide, this game has a particularly simple mixed-strategy solution, but in practice it is easiest to evaluate it numerically using backward induction.

\section{Probability of Reaching the Full Count}

Define $Q(b,s)$ to be the probability that state $(3,2)$ is ever reached before termination, starting from $(b,s)$, when both players use the run--optimal strategies corresponding to the given $p$. Then
\[
  Q(3,2) = 1, \qquad Q(4,s)=0, \qquad Q(b,3)=0.
\]
For other $(b,s)$, if the batter waits with probability $x = x(b,s)$ and the pitcher throws a ball with probability $y = y(b,s)$, then:
\begin{itemize}
  \item with probability $yx$ we move to $(b+1,s)$;
  \item with probability $y(1-x) + (1-y)x + (1-y)(1-x)(1-p)$ we move to $(b,s+1)$;
  \item with probability $(1-y)(1-x)p$ we hit a home run and terminate without having reached $(3,2)$.
\end{itemize}
Hence
\[
  Q(b,s) \;=\; yx \, Q(b+1,s) \;+\; \bigl(y(1-x) + (1-y)x + (1-y)(1-x)(1-p)\bigr)\, Q(b, s+1).
\]
This recursion is evaluated in reverse order over $b$ and $s$ once $x(\cdot,\cdot)$ and $y(\cdot,\cdot)$ are known.

Finally, the desired probability for a given $p$ is
\[
  q(p) \;=\; Q(0,0).
\]

\section{Maximization over \texorpdfstring{$p$}{p}}

We evaluate $q(p)$ for $p \in (0,1]$. Numerically, $q(p)$ increases from small $p$, attains a maximum, and then decreases. The maximum occurs at
\[
  p^\* \approx 0.2269732297,
\]
and at this value we obtain
\[
  q^\* = q(p^\*) \approx 0.2959679934.
\]

Thus, the maximal probability that an at-bat, under optimal play by both batter and pitcher, reaches the full count $(3,2)$ is
\[
  \boxed{0.2959679934}.
\]

\end{document}
